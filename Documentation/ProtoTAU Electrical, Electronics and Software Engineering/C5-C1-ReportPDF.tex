% Options for packages loaded elsewhere
\PassOptionsToPackage{unicode}{hyperref}
\PassOptionsToPackage{hyphens}{url}
%
\documentclass[
  10pt,
]{article}
\usepackage{amsmath,amssymb}
\usepackage{iftex}
\ifPDFTeX
  \usepackage[T1]{fontenc}
  \usepackage[utf8]{inputenc}
  \usepackage{textcomp} % provide euro and other symbols
\else % if luatex or xetex
  \usepackage{unicode-math} % this also loads fontspec
  \defaultfontfeatures{Scale=MatchLowercase}
  \defaultfontfeatures[\rmfamily]{Ligatures=TeX,Scale=1}
\fi
\usepackage{lmodern}
\ifPDFTeX\else
  % xetex/luatex font selection
  \setmainfont[]{DejaVuSerif.ttf}
  \setsansfont[]{DejaVuSans.ttf}
  \setmonofont[]{JetBrainsMonoNerdFontMono-Regular.ttf}
  \setmathfont[]{texgyredejavu-math.otf}
\fi
% Use upquote if available, for straight quotes in verbatim environments
\IfFileExists{upquote.sty}{\usepackage{upquote}}{}
\IfFileExists{microtype.sty}{% use microtype if available
  \usepackage[]{microtype}
  \UseMicrotypeSet[protrusion]{basicmath} % disable protrusion for tt fonts
}{}
\makeatletter
\@ifundefined{KOMAClassName}{% if non-KOMA class
  \IfFileExists{parskip.sty}{%
    \usepackage{parskip}
  }{% else
    \setlength{\parindent}{0pt}
    \setlength{\parskip}{6pt plus 2pt minus 1pt}}
}{% if KOMA class
  \KOMAoptions{parskip=half}}
\makeatother
\usepackage{xcolor}
\usepackage[left=2.21cm,right=2.21cm,top=2.5cm,bottom=2.5cm]{geometry}
\setlength{\emergencystretch}{3em} % prevent overfull lines
\providecommand{\tightlist}{%
  \setlength{\itemsep}{0pt}\setlength{\parskip}{0pt}}
\setcounter{secnumdepth}{-\maxdimen} % remove section numbering
% Make \paragraph and \subparagraph free-standing
\ifx\paragraph\undefined\else
  \let\oldparagraph\paragraph
  \renewcommand{\paragraph}[1]{\oldparagraph{#1}\mbox{}}
\fi
\ifx\subparagraph\undefined\else
  \let\oldsubparagraph\subparagraph
  \renewcommand{\subparagraph}[1]{\oldsubparagraph{#1}\mbox{}}
\fi
\usepackage{fvextra}
\DefineVerbatimEnvironment{Highlighting}{Verbatim}{breaklines,commandchars=\\\{\}}
\ifLuaTeX
  \usepackage{selnolig}  % disable illegal ligatures
\fi
\IfFileExists{bookmark.sty}{\usepackage{bookmark}}{\usepackage{hyperref}}
\IfFileExists{xurl.sty}{\usepackage{xurl}}{} % add URL line breaks if available
\urlstyle{same}
\hypersetup{
  pdftitle={Project Plan for the Second Half Session},
  pdfauthor={Christine Elizabeth Koppel},
  hidelinks,
  pdfcreator={LaTeX via pandoc}}

\title{Project Plan for the Second Half Session}
\author{Christine Elizabeth Koppel}
\date{}

\begin{document}
\maketitle

\hypertarget{project-plan---imu-sensor-application-in-the-ground-based-vehicle-velocity-calculation}{%
\section{Project Plan - IMU Sensor Application in the Ground-Based
Vehicle Velocity
Calculation}\label{project-plan---imu-sensor-application-in-the-ground-based-vehicle-velocity-calculation}}

\hypertarget{project-objective-recap}{%
\subsection{Project Objective Recap}\label{project-objective-recap}}

This dissertation aims to show how feasible it could be to use and
design the purely IMU-based velocity calculation of the vehicle in use
to give the driver of the vehicle its status of speed. Here are the
following objectives in this dissertation:

\begin{itemize}
\tightlist
\item
  Review of the IMUs (Internal Measurement Units) available, and their
  capabilities, constraints and sensitivity.
\item
  Take note of and discuss the causes of the IMU errors and what could
  be done feasibly to prepare for them.
\item
  Review of the systems used to send and receive data from the IMU.
\item
  Creating the calculation algorithms and/or a filter design to
  compensate for the errors such as axis shifts, noise, misalignment and
  other parameters affecting the accuracy. Optionally providing a memory
  safe implementation of the algorithms in the embedded device.
\item
  Implementation of the design of the vehicle and gathering following
  test run data.
\item
  Discussion of the feasibility of the pure IMU-based design over the
  GPS-based, Hall-Effect sense and other velocity calculation methods.
\end{itemize}

\hypertarget{winter-term-work-plan}{%
\subsection{Winter Term Work Plan}\label{winter-term-work-plan}}

\hypertarget{preamble}{%
\subsubsection{Preamble}\label{preamble}}

As the required hardware is acquired:

\begin{itemize}
\tightlist
\item
  BNO085 and LSM6DS3TR-C IMUs
\item
  Raspberry Pi Pico W microcontrollers -
\item
  IPS and regular LCDs
\end{itemize}

With the optionals of:

\begin{itemize}
\tightlist
\item
  3D printing material
\item
  GPS module
\end{itemize}

Boilerplate software has been implemented and which could output raw
data to the microcontroller, which proves us that the IMU is working and
is able to be used in the project.

\hypertarget{software-development}{%
\subsubsection{Software Development}\label{software-development}}

The software development is the next integral part of the project, where
the IMU data is processed and the velocity is calculated. The software
development has the following consecutive objectives:

\begin{itemize}
\tightlist
\item
  Preliminary reading material of the IMU data processing (SPI and I2C
  protocols)
\item
  Implementing raw data logging into the microcontroller
\item
  Reading mathematical solutions and implementing algorithms for the
  real-time velocity calculation, and logging it.
\item
  Algorithm implementation of filtering the data noise for the embedded
  device.
\end{itemize}

With the optional, but useful objectives of:

\begin{itemize}
\tightlist
\item
  Debugger implementation for the embedded device for troubleshooting,
  reading and debugging.
\item
  Further optimisation of the algorithms for the embedded device.
\item
  Open-Source publishing of the software under the Mozilla Public
  License 2.0 or GPL 3.0.
\end{itemize}

\hypertarget{design-implementation-and-data-collection}{%
\subsubsection{Design Implementation and Data
Collection}\label{design-implementation-and-data-collection}}

The design implementation and data collection is the next integral part
of the project, where the IMU is implemented into the vehicle and the
data is collected. The design implementation and data collection has the
following consecutive objectives:

\begin{itemize}
\tightlist
\item
  Designing the vehicle to fit the IMU and the microcontroller and
  having the initial test trial.
\item
  Gathering the data from the vehicle from consecutive results, and
  fixing the design if needed.
\item
  Analysing the data and comparing it to the other velocity calculation
  methods in the same run.
\item
  Final data collection and analysis which will be used in the
  dissertation with references to past trials and lessons learned.
\end{itemize}

\hypertarget{writing---documentation-and-dissertation}{%
\subsubsection{Writing - Documentation and
Dissertation}\label{writing---documentation-and-dissertation}}

The writing of the dissertation will be done over the both software and
design development, as it is written in a flowing concurrent
progression. The writing has the following consecutive objectives:

\begin{itemize}
\tightlist
\item
  Justifying and documenting the design choices of the sensors, hardware
  and the software chosen.
\item
  Writing the documentation of the software development and design
  implementation in the personal markdown logs.
\item
  Discussing the data collected and analysing it.
\item
  Writing the dissertation with the following chapters:

  \begin{itemize}
  \tightlist
  \item
    Introduction
  \item
    Literature Review
  \item
    Methodology
  \item
    Design and Implementation
  \item
    Results and Analysis
  \item
    Conclusion
  \item
    References
  \item
    Appendices
  \end{itemize}
\item
  Proofreading and formatting to be more readable and correct
  dissertation.
\end{itemize}

\hypertarget{optional-discussion-of-winter-term-work-plan}{%
\subsection{Optional Discussion of Winter Term Work
Plan}\label{optional-discussion-of-winter-term-work-plan}}

As this is done in regard to the student society, ProtoTAU, for
engineers to design the most efficient and reliable vehicle in an
endurance competition, this dissertation has attached optional
objectives.

\hypertarget{hardware-development-and-design}{%
\subsubsection{Hardware Development and
Design}\label{hardware-development-and-design}}

This entire design will be implemented in custom designed and also
documented hardware for controlling, powering and monitoring the
vehicle. The hardware development and design has the following
consecutive objectives that will be done in parallel with the software
development and design implementation and data collection:

\begin{itemize}
\tightlist
\item
  Power management and safety design of the vehicle.
\item
  Electronics cooling and added redundancy.
\item
  Communication of the hardware electronics and the software.
\item
  Selecting and footprinting the components to the PCBs.
\item
  Total hardware assembly.
\item
  Safety features and chassis for the electronics.
\end{itemize}

\hypertarget{documentation-foundation}{%
\subsubsection{Documentation
Foundation}\label{documentation-foundation}}

The documentation foundation will be written and published in a
society/personal website, for the main project topic after the
completion of the whole design of this. All of this will be written in
standard Markdown (.md) format. The documentation foundation has the
following consecutive objectives:

\begin{itemize}
\tightlist
\item
  Writing more automated documentation updating system for the website.
\item
  Better styling of the website.
\end{itemize}

Following parts have already been achived in the current Autumn term:

\begin{itemize}
\tightlist
\item
  Having the base system of publishing the documentation in web format
  from markdown.
\item
  Publishing the documentation of the hardware development and design in
  the ProtoTAU GitHub repository.
\item
  Webhosting the documentation.
\end{itemize}

\end{document}
